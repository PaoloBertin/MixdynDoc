\chapter{Analisi della deformazione}
La prima idea di deformazione può farsi risalire a Beekman(1630) ed a J. Bernoulli (1705)
che introdussero il concetto di misura lineare della deformazione, intesa come rapporto
fra la variazione di lunghezza di una fibra materiale e la sua lunghezza iniziale.

Successivamente furono sviluppate la \emph{teoria delle deformazioni infinitesime} 
o \emph{teoria delle piccole deformazioni} grazie ad Eulero, e la \emph{teoria delle 
deformazioni finite} dovuta principalmente a Cauchy.

La prima di queste teorie, chiamata anche \emph{teoria lineare della deformazione},
può dedursi in modo diretto, oppure come caso particolare a partire dalla generale
teoria delle deformzioni finite. Di seguito si considera questo secondo approccio.
 

